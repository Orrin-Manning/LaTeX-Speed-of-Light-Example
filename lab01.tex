\documentclass[12pt]{report}

\usepackage[margin=1in]{geometry}
\usepackage[compact]{titlesec}
\usepackage{fixltx2e}

\begin{document}

\section*{Speed of Light}
Orrin Manning\\[10pt]
Texas Tech University\\[10pt]
PHYS 3301-001\\[10pt]
Date of Experiment: 2/7/2018

\subsection*{Abstract}
This experiment attempted to observe the speed of light in a laboratory setting using The Foucault Method. A laser and rotating mirror were aligned on a rail to produce a beam level to the lab bench. The beam was then reflected off the rotating mirror to a fixed mirror, which was positioned to bounce the beam back to the rotating mirror and down the rail in the opposite direction. A beam-splitter with a microscope was positioned on the rail in the path of the beam, and two lenses were used on either side to focus the beams on the splitter. A motor would have been used to drive the rotating mirror at high rev/sec. Rotational speed and displacement of the beam image under the microscope would be used to determine the speed of light using 
$c = \frac{8\pi AD^2(Rev/sec_{cw}+Rev/sec_{ccw})}{(D+B)(s'_{cw}-s'_{ccw})}$.
However, due to vertical alignment issues an image of $s$ was never achieved and no results were recorded. In order to achieve results, more lab time or more stable laser equipment would be needed.

\subsection*{Introduction/Background/Procedure}
The speed of light is one of the most important constants known to science. An accurate measurement is critical in applying Einstein's theories of general and special relativity as well as in making reasonably accurate observations of our universe on a cosmic scale. Because of the importance of the speed of light, scientists have been trying to measure its value for a very long time.\\[10pt]
Galileo was one of the first to determine the speed of light at the dawn of natural science as its discipline. He attempted to measure the speed of light in the most intuitive way possible; measuring the time delay of light bouncing back and forth between two distant points. Of course this method is not effective on a terrestrial scale, and all he determined was it was far too great to measure with this method. R{\"o}mer later attempted to measure the speed of light by observing the motion of one of Jupiter's moons as Earth moved toward and away from Jupiter. His method achieved measurement but had a lot of error based in ignorance of the distances involved at the time. Fizeau developed a measurement method that involved using a rotating cog wheel to generated brief pulses of light, which were reflected off a mirror. He could use the distance to the mirror, angular velocity of the wheel, and rate of returning pulses passing through the wheel to calculate the speed of light within a few percent of what the currently accepted value is. Lastly, Foucault modified Fizeau's method and created a method that got him within 4 significant figures of the currently accepted value. Foucault's method is what was used in this experiment.[1]\\[10pt]
The Foucault method bounces a beam off of a rapidly rotating mirror(M\textsubscript{R}) towards a fixed mirror(M\textsubscript{F}) that then returns the beam back to the rotating mirror. During the time light travels to and from M\textsubscript{F}, M\textsubscript{R} will have rotated, creating a beam deflection on the way back to the laser. By measuring the deflection as well as distance from M\textsubscript{R} and M\textsubscript{F} and distance from an array of lenses used to focus the beam, speed of light can be found with the equation:
$$c=\frac{4AD^2\omega}{(D+B)\Delta s'}$$
See symbol meanings below procedure.\\[10pt]
Procedure:
\begin{enumerate}
	\item Start with a ruled 1m magnetic rail mounted on a flat workbench. Ensure that there are 2-15m behind the side marked 1m.
	\item Mount the laser on the side marked 1m, pointed straight down the length of the rail.
	\item Mount the rotating mirror(M\textsubscript{R}) assembly on the opposite side of the rail, facing the laser. The front of the assembly should be at the 17cm mark. 
	\item Using 2 jigs, align the laser so it is striking the center of the rotating mirror. The laser should go through both holes in the jigs and still strike the mirror. Paper can be placed under the laser to adjust height and pitch.
	\item Adjust the height and pitch of M\textsubscript{R} so the laser returns through the holes in the jigs and strikes the source of the laser.
	\item Remove jigs and mount 48mm focal length lens(L\textsubscript{1}) on the rail centered at the 93.0cm mark. Position L\textsubscript{1} so that the beam strikes the center of M\textsubscript{R}.
	\item Mount a 252mm focal length lens(L\textsubscript{2}) on the rail centered at the 62.2cm mark. Adjust L\textsubscript{2} so the beam is centered on M\textsubscript{R}.
	\item Mount a measuring microscope with the left edge at the 82.0cm mark. Position it so the beam is centered on the beam splitter, the beam splitter lever is on the same side as the metric scale, and the lever is in the down position.
	\item Place fixed mirror(M\textsubscript{F}) 2-15m in front of M\textsubscript{R}, at an angle about 12 degrees to the rail.
	\item Rotate M\textsubscript{R} so the beam is pointing at M\textsubscript{F}. Adjust the position of M\textsubscript{F} and the rotation of M\textsubscript{R} as needed until the beam hits M\textsubscript{F}.
	\item With a piece of paper on the surface of M\textsubscript{F}, adjust the position of L\textsubscript{2} to focus the beam as small as possible on M\textsubscript{F}.
	\item Use the alignment screws on the back of M\textsubscript{F} to reflect the beam back to the center of M\textsubscript{R}.
	\item Set the polarizer so the 2 dials are at an 90 degree angle from one another, and position it between directly in front of the laser. Rotate one dial until an image becomes visible in the microscope. 
	\item Adjust the scope of the microscope until you see the point image through the eyepiece. You can test if the point is correct by blocking the beam between M\textsubscript{R} and M\textsubscript{F}. If the point disappears, it is the correct image.
	\item Ensure the locking screw is completely loose on M\textsubscript{R}. Set the switch on the motor to to CW, and turn it on. Warm the motor up at 600 revolutions per second for 3 minutes.
	\item Slowly increase rotational speed until about 1,000 r/s. Push and hold the MAX REV/SEC button. Rotate the micrometer knob on the microscope until the beam point is aligned with the cross hair. Record  the speed of the motor and the micrometer reading.
	\item Reverse the direction of the motor to CCW, then repeat step 16 to take measurements.
\end{enumerate}
Once measurements have been made, the speed of light is calculated with the following equation:
$$c=\frac{8\pi AD^2(Rev/sec_{cw}+Rev/Sec_{ccw})}{(D+B)(s'_{cw}-s'_{ccw})}$$
where
\begin{itemize}
\item A = distance between L\textsubscript{2} and L\textsubscript{1} - focal length of L\textsubscript{1}
\item B = distance between L\textsubscript{2} and M\textsubscript{R}
\item D = distance between M\textsubscript{R} and M\textsubscript{F}
\end{itemize}

\subsection*{Results/Discussion}
No measurements were able to be taken during this experiment. The key source of problems was getting the right vertical alignment of the beam coming from the source going to M\textsubscript{R}. The laser used for the experiment was not the original laser intended for use with the rail and mirror set. It lacked the proper height and magnetic fasteners to achieve and maintain proper laser alignment. In an attempt to make the best of what was available, pieces of paper were used as a shim to both increase the vertical height and supply more friction to the laser. Different rotational orientations of the laser were also tested. Light weights were also applied to the laser in the form of miscellaneous lab supplies(rolls of tape, block of wood, etc.) to help steady the laser over time. However, the weight of the heavy cord pulling on the laser was enough to pull the laser off by enough to make measurement impossible. The observed result was the beam returning from M\textsubscript{R} and M\textsubscript{F} being either higher or lower on the beam-splitter than the beam coming from the source. No change in the microscope could be observed when blocking path of the laser beyond the splitter. Initial beam alignment is very important for being able to get a measurement in this experiment.\\[10pt]

\subsection*{Conclusion}
Failure to achieve results in this experiment is perhaps indicative of the illusive nature of measuring the speed of light. A proper vertical alignment of the laser traveling to M\textsubscript{R} can be difficult to achieve, and renders all work after that point unproductive if it is not correct. This was the biggest source of error in practice. Other sources of error that can arise in this experiment are improper position and focus of the lenses, malfunction of the motor, and poor technique in handling the micrometer dial on the microscope.\\[10pt]
On a future attempt at this experiment, 2 things could be improved to have a better chance at producing results. Firstly, with more time spent in the lab, good laser alignment could be achieved with enough trial and error. Secondly, having a laser better suited for use in this experiment would make the setup for this experiment much easier and faster.\\[10pt]
This experiment and experiments like it are critically important to physics as a whole. The constant speed of light is one of the most significant properties of our universe, and having an accurate measure of its value is important for being able to use special and general relativity, and for being able to observe and explore our universe on a cosmic scale. 

\subsection*{Reference/Work Cited}
[1] PASCO. \textit{Speed of Light}. (2/7/18)

\end{document}\grid
